\documentclass[../main.tex]{subfiles}

\begin{document}

\section{Conclusion}
\label{sec: Conclusion}

This thesis proposed a human-machine ensemble method for economic forecasts.
Short-term economic forecasts are difficult, so various methods have been proposed so far.
Machine learning methods such as RNN create a prediction model based mainly on past time series.
In surveys for economic forecasts, individual respondents make forecasts considering various information such as policy and economy.
Thus, whereas machines are good at forecasts that depends on the past patterns, humans can make forecasts flexibly even when changes not in the past such as financial crisis.
However, traditional ensemble methods can not utilize these difference.
Therefore, we modeled a machine that outputs expected squared error depending on input by introducing posterior distributions to the output of the machine.
Moreover, based on this model, we proposed an ensemble method to dynamically change the combination of the machine and humans depending on input.
The ensemble minimizes the expected error.

In addition, we conducted an experiment to apply the proposed method to actual inflation forecasts.
The objectives were model verification, evaluation of forecast accuracy and confirmation of the behavior of the proposed method.
We created RNN prediction models that output discrete probability distributions for machine forecasts, and used two surveys for human forecasts.
From these combinations, we constructed seven ensembles.
Evaluation was performed using test set prepared separately from the training set.
We obtained following three results.
First, the proposed models were applicable to actual inflation forecasts.
Second, in 4 out of 7 cases, the ensemble made more accurate forecasts than the machine only and the humans only.
Finally, we showed that the ensemble behaved as expected.
When machine received a sequence not in the past, the ensemble drew human forecasts.

The proposed ensemble method improved forecast accuracy in the majority cases, but not always reduce the error.
The results suggest that the difference of a machine and humans is important for the proposed method to perform well.
It is necessary that the group of machines and humans is more diverse than the group of humans only.
In the experiment, machines and humans did not make so different forecasts since they are influenced by the index at the time of forecasting.
However, there is a possibility of a more efficient human-machine ensemble method by extending the models or improving machine forecasts.

Finally, the main contributions of this thesis are as follows:

\subparagraph{Modeling forecasts by humans and a machine and inventing an ensemble method using them}
We modeled the case where the expected error changes with each forecast by assuming the posterior distribution to the outputs of machines.
In addition, we proposed a method which determines the optimal number of humans dynamically depending on the expected error of a machine forecast.

\subparagraph{Implementation and evaluation of economic forecasts using the proposed ensemble method}
We applied the human-machine ensemble method to inflation forecasts.
As a result, we confirmed the proposed model can be applied to real inflation forecasts and forecast accuracy improves with 4 out of 7 data.

\end{document}
